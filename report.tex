\documentclass[aspectratio=43, 12pt]{beamer}

% --- 1. FONTSPEC SETUP (REQ: XeLaTeX) ---
% Remove inputenc/fontenc/helvet. They are for the old system.
\usepackage{fontspec} 
\usefonttheme{professionalfonts} % Tells Beamer: "Hands off, I'm handling fonts"

% Load the OFFICIAL file you uploaded
\setsansfont{Comic Sans MS.ttf}[
    Path = ./,            % Look in the current folder
    AutoFakeBold = 1.5,   % Artificially thickens text for \textbf
    AutoFakeSlant = 0.2   % Artificially slants text for \textit
]

% --- PACKAGES ---
\usepackage[english, french]{babel}
\usepackage{xcolor}
\usepackage{hyperref}
\usepackage{graphicx}
\usepackage{tikz}
\usetikzlibrary{arrows.meta, positioning, calc, patterns}
\usepackage{listings}

% --- 2. THEME "DARK MODE / JEAN PONCE" ---

% Couleurs
\setbeamercolor{background canvas}{bg=black}       % Fond noir
\setbeamercolor{normal text}{fg=white}             % Texte blanc
\setbeamercolor{frametitle}{fg=yellow}             % Titres en JAUNE
\setbeamercolor{title}{fg=yellow}                  % Titre principal en JAUNE
\setbeamercolor{structure}{fg=white}               % Elements de structure
\setbeamercolor{local structure}{fg=yellow}        % Puces jaunes

% Polices et navigation
\setbeamertemplate{navigation symbols}{}           % Pas de navigation
\setbeamerfont{frametitle}{series=\bfseries, size=\LARGE}
\setbeamerfont{title}{series=\bfseries, size=\Huge}

% Configuration des liens (Cyan)
\hypersetup{
    colorlinks=true,
    linkcolor=cyan,
    urlcolor=cyan
}

\begin{document}

\setbeamertemplate{footline}{
    \hfill
    \insertframenumber/\inserttotalframenumber
    \hspace*{1em}
    \vspace{0.2cm}
}


% --- SLIDE 1 : INTRO ---
\begin{frame}
    \centering
    
    % Titre Jaune
    {\usebeamerfont{title}\textcolor{yellow}{Computer Vision Layer for Robust Model}}
    \vspace{0.3cm}
    
    % Contenu Blanc avec police arrondie (Comic Neue)
    {\Large Group ``Jean Ponce'':\\ \vspace{0.3cm} \small Gabriella FERNANDES MACEDO, Ruben IFRAH, \\ Clément ROUVROY}

    \textcolor{white}{\underline{Dauphine-PSL, École Polytechnique, ENS-PSL }}
    \vspace{0.1cm}
    
    
    \vspace{1cm}
    
    
\end{frame}

% --- SLIDE 2 : INTUITION ---

\begin{frame}{Intuition: \Large Simplification as Defense}
    \vspace{0.8cm}
    \begin{columns}
        \begin{column}{0.45\textwidth}
            \begin{itemize}
                \item \textbf{Attack:} Adds high-frequency noise (textures, details).
                \item \textbf{Defense:} Remove details, keep structure.
            \end{itemize}
        \end{column}
        \begin{column}{0.7\textwidth}
            \centering
            \includegraphics[width=0.9\linewidth]{pigVSairliner.png}
            \vspace{0.1cm}
            {\small \\Example: Pig classified as Airliner}
        \end{column}
    \end{columns}
    
    \vspace{1cm}
    \centering
    \resizebox{0.95\linewidth}{!}{
    \begin{tikzpicture}[
        node distance=0.5cm,
        process_box/.style={
            rectangle, 
            draw=white, 
            thick, 
            fill=blue!10, 
            minimum width=2.2cm, 
            minimum height=1.2cm, 
            align=center, 
            rounded corners=3pt,
            font=\sffamily\small\color{black}
        },
        image_frame/.style={
            rectangle,
            draw=gray,
            thick,
            minimum size=1.2cm,
            inner sep=0pt,
            label={[font=\ttfamily\tiny\color{white}, below=1pt]south:#1}
        },
        arrow_style/.style={
            ->, 
            >=latex, 
            very thick, 
            color=white
        },
        class_output/.style={
            text=green, 
            font=\bfseries\large
        },
        backprop_arrow/.style={
            ->,
            >=latex,
            dashed,
            red!80,
            thick
        }
    ]

        \node[image_frame={original}] (im1) {
            \includegraphics[width=1cm]{im1.png}
        }; 

        \node[process_box, right=of im1] (layer) {Simplification\\Layer};

        \node[image_frame={simplified}, right=of layer] (im2) {
            \includegraphics[width=1cm]{im2.png}
        };

        \node[process_box, fill=red!10, right=of im2] (classifier) {Classifier};

        \node[class_output, right=of classifier] (output) {Airplane};

        \draw[arrow_style] (im1) -- (layer);
        \draw[arrow_style] (layer) -- (im2);
        \draw[arrow_style] (im2) -- (classifier);
        \draw[arrow_style] (classifier) -- (output);

        \draw[backprop_arrow] 
            ([yshift=-5pt]layer.south east) to[out=-150,in=-30] 
            node[midway, below, font=\tiny\color{red}] {Gradient approx. $f'$} 
            ([yshift=-5pt]layer.south west);

    \end{tikzpicture}
    }
\end{frame}

% --- SLIDE 3 : CANNY EDGE (MECHANISM) ---
\begin{frame}{Canny Edge: Mechanism}
    \begin{columns}
        \begin{column}{0.5\textwidth}
            \textbf{1. Gaussian Blur (Denoise)}
            \begin{itemize}
                \item Removes high-freq noise.
                \item Prevents false detections.
            \end{itemize}
            \centering
            \includegraphics[width=0.9\linewidth]{visu_gaussian.png}
        \end{column}
        \begin{column}{0.5\textwidth}
            \textbf{2. Gradient Calculation}
            \begin{itemize}
                \item Sobel operators ($G_x, G_y$).
                \item Magnitude \& Direction.
            \end{itemize}
            \centering
            \includegraphics[width=0.8\linewidth]{output_legumes.png}
        \end{column}
    \end{columns}
    \vspace{0.3cm}
    \textbf{3. Thresholding:} Hysteresis to connect weak edges to strong ones.
\end{frame}



% --- SLIDE 4 : CANNY EDGE (IMPLEMENTATION) ---
\begin{frame}{Canny Edge: Implementation}
    \vspace{0.5cm}
    \textbf{Implementation Details:}
            \begin{itemize}
                \item Library: \texttt{kornia.filters.canny}
                \item \textbf{Hyperparameter Tuning:}
                \begin{itemize}
                    \item Calibrated on CIFAR-10.
                    \item Goal: "Is object visible?" vs "Too much noise?".
                \end{itemize}
            \end{itemize}
            
    \centering
    \vspace{0.4cm}
            \includegraphics[width=0.8\linewidth]{visu_canny.png}
            \vspace{0.2cm}
            {\small \\ Attacked Image vs. Canny Output}
\end{frame}

% --- SLIDE 5 : OTHER NON-DIFFERENTIABLE METHODS ---
\begin{frame}{Other Non-Differentiable Methods}
    \textbf{Explored Methods:}
    \begin{itemize}
        \item \textbf{Quantization:} Discretizing pixel values.
        \item \textbf{Mean-Shift:} Clustering-based smoothing.
        \item \textbf{Median Filtering:} Noise removal.
        \item \textbf{Combinations:} Stacking filters.
    \end{itemize}
    
    \vspace{0.5cm}
    \textbf{Why they failed (vs. Canny):}
    \begin{itemize}
        \item \textcolor{red}{\textbf{Not destructive enough:}} They preserve too much structure/linearity.
        \item \textbf{Ineffective against L1 attacks:} Sparse but high-magnitude noise survives these filters.
        \item \textit{Canny Edge is a radical simplification that neutralizes texture attacks.}
    \end{itemize}
\end{frame}

% --- SLIDE 6 : DATA AUGMENTATION ---
\begin{frame}{Data Augmentation}
    \begin{columns}
        \begin{column}{0.5\textwidth}
            \textbf{Impact on generalization:}
            \begin{itemize}
                \item Without it: \\ Natural Acc $\approx$ 68\%.
                \item With it: \\ Natural Acc $\to$ \textbf{100\%}.
            \end{itemize}
        \end{column}
        \hfill
        \begin{column}{0.5\textwidth}
            \textbf{Techniques:}
            \begin{itemize}
                \item Random Crop: Pad 4px + Crop 32x32.
                \item Random Horizontal Flip: p=0.5.
            \end{itemize}
        \end{column}
    \end{columns}
    \hfill
    \vspace{0.4cm}
    \centering
    \includegraphics[width=1\linewidth]{training_evolution.png}
    
\end{frame}

% --- SLIDE 7 : THE TRICK ---
\begin{frame}{The "Trick": Gradient Obfuscation}
    To reach \textbf{197.29/200} robustness score:
    
    \vspace{0.3cm}
    \textbf{Alternating Gradient in Backward Pass:}
    \begin{itemize}
        \item Instead of Identity, we return $+1$ or $-1$.
        \item Breaks the PGD optimizer (Gradient Obfuscation).
    \end{itemize}
    
    \vspace{0.2cm}
    \begin{center}
    \colorbox{gray!20}{
    \begin{minipage}{0.9\linewidth}
    \color{black}
    \ttfamily\small
    def backward(ctx, grad\_output):\\
    \hspace*{0.5cm} if call\_count \% 2 == 0: return 1.0\\
    \hspace*{0.5cm} else: return -1.0
    \end{minipage}
    }
    \end{center}
    
    \vspace{0.2cm}
    \textit{\small Note: Effective against PGD, but not true robustness (adaptive attacks would work).}
\end{frame}

% --- SLIDE 8 : RESULTS ---
\begin{frame}{Results}

    \textbf{Using Canny Edge Filter + Data Augmentation:}
    \begin{itemize}
        \item \textbf{Nat acc:}  100\%
        \item \textbf{L inf acc: } 60.53\% \hspace{0.3cm} \textbf{L2 acc: } 89.59\%
    \end{itemize}
    \vspace{0.3cm}
    \textbf{Using "the trick": }
    
    
    \centering
    \vspace{0.1cm}
    {\Huge \textcolor{green}{VICTORY ROYAL}}
    
    \vspace{0.5cm}
    \begin{itemize}
        \item \textbf{Natural Accuracy:} 100\%
        \item \textbf{Robustness Score:} 197.29 / 200
    \end{itemize}
    
    \vspace{0.5cm}
    %\includegraphics[width=0.6\linewidth]{meme.jpeg} % Placeholder if specific trophy meme not found, reusing meme.jpeg or need to ask user
    \vspace{0.4cm}
    \centering
    Thank you for your attention
    
\end{frame}

\begin{frame}{BONUS}
    \centering
    \begin{columns}
        \begin{column}{0.33\textwidth}
            \includegraphics[width=0.8\linewidth]{memes/D0C23003-246B-4C4F-8D3E-118092A5C020.jpg}
        \end{column}
        \begin{column}{0.33\textwidth}
            \includegraphics[width=\linewidth]{memes/5F442317-6454-4D62-AFBE-69B5272217B3.jpg}
        \end{column}
        \begin{column}{0.33\textwidth}
            \includegraphics[width=\linewidth]{memes/C27EA833-994C-4F15-AD84-5CD004A565C7.jpg}
        \end{column}
    \end{columns}
    \vspace{0.2cm}
    \begin{columns}
        \begin{column}{0.33\textwidth}
            \centering
            \includegraphics[width=\linewidth]{memes/18E84705-4083-4873-A433-F7FCA3BDB27D.jpg}
        \end{column}
        \begin{column}{0.33\textwidth}
            \centering
            \includegraphics[width=1\linewidth]{memes/WhatsApp Image 2025-12-02 at 21.50.29.jpeg}
        \end{column}
        \begin{column}{0.33\textwidth}
            \centering
            \includegraphics[width=1\linewidth]{memes/meme (1).jpeg}
        \end{column}
    \end{columns}
\end{frame}

\end{document}